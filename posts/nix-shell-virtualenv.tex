---
title: nix-shell is like virtualenv for everything
date: 2017-06-07
---

\begin{document}
\textit{nix-shell} can be use to make isolated environment with desired package installed. 

It is better than \textit{installed everything globally} which might lead to \textit{dependency conflicted}.
You should only installed what you need.

I've try \textit{nix-shell} with \textit{ghc} and \textit{python} and it work.

The simple use case is just pass \textit{-i <packages>} to \textit{nix-shell} and you have python available to use.

\begin{minted}{shell}
$ nix-shell -i python
[nix-shell:]$ python --version
Python 2.7.13

$ nix-shell -p python3
[nix-shell:]$ python --version
Python 3.6.1
\end{minted}

GHC example:

\begin{minted}{shell}
$ nix-shell -i python
[nix-shell:]$ python --version
Python 2.7.13

$ nix-shell -p python3
[nix-shell:]$ python --version
Python 3.6.1
\end{minted}


\textit{nix-shell} configuration is done via \textit{shell.nix} or \textit{default.nix}

Examples: \href{../nix-vim-haskell}{Haskell}, and \href{../nix-vim-python}{Python}

\end{document}


